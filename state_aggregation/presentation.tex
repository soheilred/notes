% @Author: soheilred
% @Date:   2018-10-28 01:10:54
% @Last Modified by:   soheilred
% @Last Modified time: 2019-02-18 15:24:39
\documentclass{beamer}

\let\val\undefined
\usepackage{pgf}
\usepackage{pgfplots}
\usepackage{tikz}
\usepackage{booktabs}
\usepackage{natbib}
\usepackage{framed}
\usepackage{longtable}
\usepackage{bigdelim,multirow}
\usepackage{amsmath}
\usepackage{amsthm}
\usepackage{mathtools}


\usetikzlibrary{arrows,automata,backgrounds,positioning,decorations,intersections,matrix}

% *** Styles ***
\setbeamertemplate{navigation symbols}{}
\usecolortheme{dolphin}
%\usecolortheme{rose}
\setbeamercovered{transparent}
\usefonttheme{professionalfonts}
%\usefonttheme[onlymath]{serif}

% 
\addtobeamertemplate{navigation symbols}{}{%
    \usebeamerfont{footline} %
    \usebeamercolor[fg]{footline}%
    \hspace{1em}%
    \insertframenumber/\inserttotalframenumber
}

\DeclarePairedDelimiter{\norm}{\lVert}{\rVert}
\DeclarePairedDelimiter\abs{\lvert}{\rvert}%


% \setlist[itemize,1]{label=$\times$}
% \setlist[itemize,2]{label=$\checkmark$}
% \setlist[itemize,3]{label=$\diamond$}
% \setlist[itemize,4]{label=$\bullet$}


% *** Colors ***
\newcommand{\tc}[2]{\textcolor{#1}{#2}}
\newcommand{\tcb}[1]{\tc{blue}{#1}}
\newcommand{\tcr}[1]{\tc{red}{#1}}
\newcommand{\tcg}[1]{\tc{green}{#1}}

\def\checkmark{\tikz\fill[scale=0.4](0,.35) -- (.25,0) -- (1,.7) -- (.25,.15) -- cycle;} 

\newcommand{\Ex}{\mathbb{E}}
%\newcommand{\Pr}{\mathbb{P}}
\DeclareMathOperator{\Var}{Var}

\definecolor{varcolor}{RGB}{132,23,49}
\newcommand{\varname}[1]{\textcolor{varcolor}{\mathsf{#1}}}

\title{Performance Loss Bounds for Approximate Value Iteration with State Aggregation \\ {\tiny Mathematics of Operations Research}}
\author{Van Roy}
\date{\today}

\begin{document}
\begin{frame}
	\maketitle

\end{frame}
%=====================================%
\begin{frame}
	\frametitle{Problem Definition}
	\begin{itemize}
		\item What is this paper about?

		How to approximate $J^*$, the optimal cost-to-go function (value function), with $\Phi r$?
		\begin{itemize}
			\item State Aggregation
			\item Value Iteration
			\item In particular, how to find $r \in \Re^K$?
		\end{itemize}
		\item What is this paper NOT about?

		How to partition the space (or find $\Phi \in \Re^{|\varphi| \times K}$)?
	\end{itemize}

	\footnotetext{$|\varphi|$ is the number of states}
\end{frame}
%=====================================%

%=====================================%

\begin{frame}
	\frametitle{Approximating $J$}
	\begin{itemize}
		\item 
		\begin{equation} \label{eq:min}
			\min\limits_{r} \norm{J - \Phi r}_{2,\pi}
		\end{equation}
		\item Projection with respect to a weighted Euclidean norm $\norm{.}_\pi$
		\[
			\norm{J}_{2,\pi} = \left( \sum_{x \in \varphi}{\pi (x) J^2(x)} \right)^{1/2}
		\]
		$\pi \in \Re_{+}^{\abs{\varphi}}$ is a vector of weights, showing the \tcr{\emph{importance}} of each state
		\item To get \textit{\LARGE \tcb{r}}, we need to run value iteration
		\[
			\Phi r^{(l+1)} = \Pi_{\pi} T \Phi r^{(l)}
		\]
	\end{itemize}
\end{frame}
%=====================================%

%=====================================%

\begin{frame}
	\frametitle{Calculating $r$}
		\[
			\Phi r^{(l+1)} = \Pi_{\pi} T \Phi r^{(l)}
		\]
	\begin{itemize}
		\item T is the dynamic programming operator
		\item T is a contraction
		\item $\Pi_\pi$ is max-norm nonexpansive
		\item $\Pi_\pi$ operator (matrix) projects onto the column space of $\Phi$ with respect to a weighted Euclidean Norm that minimizes equation (\ref{eq:min})
		\item 
		% \[
		% 	(\Pi_\pi J) (x) = \frac{\sum_{y \in \varphi_k}{\pi(y) J(y)}}{\sum_{y \in \varphi_k}{\pi(y)}}
		% \]
		\[
			\Pi_\pi = \Phi(\Phi^T D \Phi)^{-1} \Phi^T D
		\]
		where $D=diag(\pi_i)$
	\end{itemize}
\end{frame}
%=====================================%

%=====================================%

\begin{frame}
	\frametitle{Bounds}
	\begin{itemize}
		\item \tcr{Without considering} the importance weights, and when $\mu$ is greedy with respect to $\tilde{J}$ we have the bound\footnote{$\norm{J_\mu - J^*}_{\infty}$ is performance loss}:
		\begin{equation}
			\norm{J_\mu - J^*}_{\infty} \leq \frac{2\alpha}{1-\alpha}\norm{J^* - \tilde{J}}_{\infty}
		\end{equation}
		\item \tcr{Considering} the importance weights and when $\Phi \tilde{r} = \Pi_{\pi} T \Phi \tilde{r}$
		\begin{itemize}
			\item \emph{Approximation Error Bound}
			\[
				\norm{\Phi \tilde{r} - J^*}_{\infty} \leq \frac{2}{1-\alpha} \min\limits_{r \in \Re^K}\norm{J^* - \tilde{J}}_{\infty}
			\]
			\item \emph{Performance Loss Bound}
			\[
				(1-\alpha)\norm{J_{\mu \tilde{r}} - J^*}_{\infty} \leq \frac{4\alpha}{1-\alpha}\min\limits_{r \in \Re^K}\norm{J^* - \Phi r}_{\infty}
			\]
			\item HOWEVER!!!
		\end{itemize}
	\end{itemize}
\end{frame}
%=====================================%

%=====================================%

\begin{frame}
	\frametitle{Using the Invariant Distribution}
	\begin{itemize}
		\item if $\pi_{\tilde{r}}$ is is the invariant state distribution of transition matrix $P_{\mu_{\tilde{r}}}$ \footnote{$\pi$ is an invarient of P if $\pi^T P = \pi^T$}
		\[
			(1-\alpha)\pi_{\tilde{r}}^T(J_{\mu \tilde{r}} - J^*) \leq 2\alpha \min\limits_{r \in \Re^K}\norm{J^* - \Phi r}_{\infty}
		\]
		\item Compare it with:
		\[
			(1-\alpha)\norm{J_{\mu \tilde{r}} - J^*}_{\infty} \leq \frac{4\alpha}{1-\alpha}\min\limits_{r \in \Re^K}\norm{J^* - \Phi r}_{\infty}
		\]
		\item Weighting Euclidean Norm projection by the invariant distribution of a greedy (or $\epsilon$-greedy) policy improves the performance loss.
	\end{itemize}
\end{frame}
%=====================================%

%=====================================%

\begin{frame}
	\begin{center}
		\Huge Thank You!
	\end{center}
\end{frame}
%=====================================%


% %=====================================%

% \begin{frame}
% 	\frametitle{Background}
% 	\begin{itemize}

% 	\end{itemize}
% \end{frame}
% %=====================================%



%=====================================%

% \begin{frame}
% 	\begin{center}
% 		\Huge Thank You!
% 	\end{center}
% \end{frame}
%=====================================%

% \begin{frame}
	% \frametitle{Background}
% 	\begin{itemize}
% 		\item 
% 	\end{itemize}
% \end{frame}


\end{document}
