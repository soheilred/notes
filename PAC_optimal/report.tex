\documentclass{article}
%\documentclass{IEEETran}

%\usepackage{anysize}
%\usepackage[left=1.9cm,right=1.9cm,top=2.54cm,bottom=1.9cm]{geometry}
%\usepackage{sectsty}
%\sectionfont{\normalsize \bf}
%\subsectionfont{\normalsize \bf}

%\IEEEoverridecommandlockouts  
%\overrideIEEEmargins

%\usepackage{siunitx}
\usepackage{setspace} 
%\doublespacing
\usepackage{cite}
\usepackage{amsmath}
\usepackage{amsthm}
\theoremstyle{remark}
\newtheorem{assumption}{Assumption}
\theoremstyle{remark}
\newtheorem{remark}{Remark}
\theoremstyle{remark}
\newtheorem{theorem}{Theorem}
\theoremstyle{remark}
\newtheorem{lemma}{Lemma}
\theoremstyle{remark}
\newtheorem{property}{Property}
\theoremstyle{remark}
\newtheorem{definition}{Definition}
\usepackage{graphicx}
\usepackage{fancyhdr}
\usepackage[bottom]{footmisc}
\usepackage{hyperref}
\usepackage{amssymb}
\usepackage{enumerate}
%===================================

\title{PAC Optimal MDP Planning with Application to Invasive Species Management}

\makeatletter
\let\@fnsymbol\@arabic
\makeatother
% \author{Soheil}
\date{}
\begin{document}
\maketitle
\section{Main Ideas}
\begin{itemize}
	\item The idea of MDPs are very well-applicable in the realm of environmental managment and in particular, invasive species.
	\item As discussed before, in case of big MDPs, we have no other choices rather than using samples to solve the MDP.
	\item In this paper, another approach is studies. Instead of batch of samples, they have a simulator, which is able to produce a random outcome of the process in each state-action pair.
	\item The paper bases the idea of PAC-RL on Fiechter's work. However, it extends the Hoeffding-bound confidence intervals to a multinomial CI, that was introduced in MBIE.
	\item In order to find a tighter CI for the transition probability, a method called Good-Turning estimate of the missing mass is used. This is done based on the fact that transition probabilities are usually sparse in real world, meaning that they are limitted (and often a couple of) $s'$ that the MDP can reach out to from the state $s$.
\end{itemize}
% \section{Questions}

\end{document}


%===============================
%========== NOT TO RUN =========
\iffalse


\usepackage{titlesec}

\titleformat{\section}
  {\normalfont\Large\bfseries}   % The style of the section title
  {}                             % a prefix
  {0pt}                          % How much space exists between the prefix and the title
  {Section \thesection:\quad}    % How the section is represented

% Starred variant
\titleformat{name=\section,numberless}
  {\normalfont\Large\bfseries}
  {}
  {0pt}
  {}
  

asdasd \cite{1013341}. Fig.~\ref{Fig_Example}.

{\fontfamily{pcr}\selectfont 
run files}

\begin{enumerate}
  \item The labels consists of sequential numbers.
  \item The numbers starts at 1 with every call to the enumerate environment.
\end{enumerate}

\begin{itemize}

\end{itemize}

======== HYPERLINK =============
Find the \texttt{run files} for all variants in \href{https://github.com/SHi-ON/InfoRet/tree/master/results/Assignment_4}{here}


====== FIGURES ========

\begin{figure}
	\centering
	\includegraphics[scale=.40]{Fig_Example.pdf}
	\caption{Example.}
	\label{Fig_Example}
\end{figure}

\input{second}

========= EQUATIONS =============

\begin{equation}
\left\|\frac{\partial V}{\partial\mathbf{x}_1}\right\|\left\|\mathbf{x}_1\right\|\leq c_1V\quad \text{for}\; \left\|\mathbf{x}_1\right\|\geq c_2
\end{equation}


============ TABLES =============

\begin{tabular}{ |p{3cm}|p{3cm}|p{3cm}|  }
\hline
\multicolumn{3}{|c|}{Country List} \\
\hline
Country Name     or Area Name& ISO ALPHA 2 Code &ISO ALPHA 3 \\
\hline
Afghanistan & AF &AFG \\
Aland Islands & AX   & ALA \\
Albania &AL & ALB \\
Algeria    &DZ & DZA \\
American Samoa & AS & ASM \\
Andorra & AD & AND   \\
Angola & AO & AGO \\
\hline
\end{tabular}

\begin{center}
\begin{tabular}{ |c|c|c| } 
 \hline
 cell1 & cell2 & cell3 \\ 
 cell4 & cell5 & cell6 \\ 
 cell7 & cell8 & cell9 \\ 
 \hline
\end{tabular}
\end{center}




\begin{lstlisting}[language=Python, caption=Code's name]
import numpy as np
 
def incmatrix(genl1,genl2):
    m = len(genl1)
    n = len(genl2)
    M = None #to become the incidence matrix
    VT = np.zeros((n*m,1), int)  #dummy variable
\end{lstlisting}



\begin{verbatim}
Put your codes here!
\end{verbatim}
\fi
